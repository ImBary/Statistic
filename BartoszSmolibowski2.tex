\documentclass{article}
\usepackage{graphicx}
\usepackage{amsmath}
\usepackage{amssymb}
\usepackage{subcaption}


\begin{document}

\title{Analiza cen spolek 2}
\author{Bartosz Smolibowski}
\date{November 2024}

\maketitle

\begin{enumerate}
    \item \textbf{Temat: }Analiza cen spolki
    \item \textbf{Nazwa  spolki: } Pepco, Energa
    \item \textbf{Okres danych: }01.01.2022 - 31.12.2022
\end{enumerate}

\vspace{12pt}

\section{Wykres rozrzutu spolek z histogramami brzegowymi}

\begin{figure}[h]
    \centering
    \includegraphics[width=0.5\linewidth]{Rplot06.png}
    
\end{figure}

\clearpage 

\section{srednie $\mu$, kowariancji, wspólczynnik korelacji, macierz kowariancji $\sum$,macierz korelacji P}

\vspace{12pt}

\textbf{wykres srednich $\mu$ }

\begin{table}[h]
    \centering
    \begin{tabular}{lc}
        \toprule
        \textbf{Pepco} & \textbf{Energa} \\
        \midrule
         50.16818  & 100.36012  \\
    
        \bottomrule
    \end{tabular}
    \caption{Wektor średnich $\mu$}
    \label{tab:mean_vector}
\end{table}

\vspace{30pt}

\textbf{Macierze kowariancji }

\vspace{30pt}

\begin{table}[h]
    \centering
    \begin{tabular}{lcc}
        \toprule
        & \textbf{Pepco} & \textbf{Energa} \\
        \midrule
         \textbf{Pepco} & 106.2091  & 163.0173 \\
         \textbf{Energa} & 163.0173 & 899.6912
        \bottomrule
    \end{tabular}
    \caption{Macierz kowariancji(1)}
    \label{tab:mean_vector}
\end{table}

\vspace{30pt}

\begin{table}[h]
    \centering
    \begin{tabular}{lcc}
        \toprule
        & \textbf{Pepco} & \textbf{Energa} \\
        \midrule
         \textbf{Pepco} & 106.1029 & 162.8543 \\
         \textbf{Energa} & 162.8543 & 898.7915
        \bottomrule
    \end{tabular}
    \caption{Macierz kowariancji(2)}
    \label{tab:mean_vector}
\end{table}

\newpage

\textbf{Macierz Korelacji }

\begin{table}[h]
    \centering
    \begin{tabular}{lcc}
        \toprule
        & \textbf{Pepco} & \textbf{Energa} \\
        \midrule
         \textbf{Pepco} & 1 & 0.5273589 \\
         \textbf{Energa} & 0.5273589 & 1
        \bottomrule
    \end{tabular}
    \caption{Macierz korelacji(3)}
    \label{tab:mean_vector}
\end{table}

\vspace{30pt}

\textbf{Kowariancja miedzy Pepco i Energa: } 163.0173

\vspace{30pt}

\textbf{Wpolczynnik korelacji miedzy Pepco i Energa: }0.5273589

\vspace{30pt}

\textbf{(1)Estymator: }S_{xy} = \frac{1}{n} \sum_{i=1}^{n} (X_i - \bar{X})(Y_i - \bar{Y})

\vspace{20pt}

\textbf{(2)Estymator obciazony: }S_{xy} = \frac{1}{n-1} \sum_{i=1}^{n} (X_i - \bar{X})(Y_i - \bar{Y})

\vspace{20pt}

\textbf{(3)Estymator Korelacji: } S_x^2 = \hat{\sigma}_1^2 = \frac{1}{n-1} \sum_{i=1}^{n} (X_i - \bar{X})^2\

\vspace{20pt}

\textbf{(3) Estynator Korelacji: } S_y^2 = \hat{\sigma}_1^2 = \frac{1}{n-1} \sum_{i=1}^{n} (Y_i - \bar{Y})^2\

\newpage

\section{ Gestosc rozkladu normalnego dwuwymiarowego, wzory gestosci rozkladów brzegowych}

\vspace{20pt}

\textbf{Wzor Gestosc rozkladu normalnego dwynuariwego: }

\begin{equation}
    f(x, y) = \frac{1}{{2\pi\sigma_1\sigma_2\sqrt{1 - \rho^2}}} \exp\left(-\frac{1}{2(1 - \rho^2)} \left(\frac{(x - \mu_1)^2}{\sigma_1^2} - 2\rho\frac{(x - \mu_1)(y - \mu_2)}{\sigma_1\sigma_2} + \frac{(y - \mu_2)^2}{\sigma_2^2}\right)\right)
\end{equation}

\vspace{30pt}

\textbf{Rozklady brzegowe}

\vspace{20pt}

\item Jeśli $(X, Y) \sim N(\mu_1, \mu_2, \sigma_1, \sigma_2, \rho)$, to $X \sim N(\mu_1, \sigma_1)$ i $Y \sim N(\mu_2, \sigma_2)$.

\begin{equation}
    $f_1(x) = \int_{-\infty}^{\infty} f(x, y) \, dy = \frac{1}{\sqrt{2\pi\sigma_1}} e^{-\frac{(x-\mu_1)^2}{2\sigma_1^2}}, \quad x \in \mathbb{R}$
\end{equation}

\begin{equation}
    $f_1(y) = \int_{-\infty}^{\infty} f(x, y) \, dy = \frac{1}{\sqrt{2\pi\sigma_1}} e^{-\frac{(y-\mu_1)^2}{2\sigma_1^2}}, \quad y \in \mathbb{R}$
\end{equation}

\newpage

\textbf{Wykresy brzegowe}

\vspace{30}

\begin{figure}[h]
    \centering
    \includegraphics[width=0.9\linewidth]{Rplot08.png}
    \label{fig:enter-label}
\end{figure}

\newpage

\section{Proba licznosci}

\vspace{30}

\textbf{wykres proby licznosci}

\begin{figure}[h]
    \centering
    \includegraphics[width=0.5\linewidth]{Rplot09.png}
    \caption{wykres proby licznosci}
    \label{fig:enter-label}
\end{figure}

\begin{figure}[h]
    \centering
    \includegraphics[width=0.5\linewidth]{Rplot06.png}
    \caption{wykres z wartosci spolek}    
\end{figure}


\end{document}

