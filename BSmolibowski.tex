\documentclass{article}
\usepackage{graphicx} % Required for inserting images
\usepackage{amsmath} % Wymagane do używania symboli matematycznych i równań
\usepackage{amssymb}
\usepackage{subcaption} % Required for subfigures
\begin{document}

\title{Analiza cen spółek}

\author{Bartek Smolibowski}

\date{December 2023}

\begin{enumerate}
    \item \textbf{Temat: }Analiza cen spolki
    \item \textbf{Nazwa spolki:} Grupa Azoty Police SA (PCE)
    \item \textbf{Okres danych:} 01.01.2022 - 31.12.2022
\end{enumerate}

\textbf{Spolka Grupa Azoty Zakłady Chemiczne "Police" SA} to polskie przedsiebiorstwo branzy wielkiej syntezy chemicznej.

\vspace{12pt} % Adds a vertical space of 12 points (adjust as needed)

\textbf{Kursy zamkniec na przestrzeni czasu.}

\vspace{12pt}

\textbf{Wykres:}
\begin{figure}[h]
    \centering
    \includegraphics[width=0.7\linewidth]{Rplot.png}
    \caption{Wykres kursow zamkniec}
    \label{fig:enter-label}
\end{figure}

\textbf{Histogram:}

\vspace{12pt}

\begin{figure}[h]
    \centering
    \includegraphics[width=0.7\linewidth]{Rplot01.png}
    \caption{Histogram kursow zamkniec}
    \label{fig:enter-label2}
\end{figure}

\hrulefill

\begin{itemize}
    \item[$\bullet$] \textbf{Skośność} liczona wzorem $\text{SKE} = \frac{n \sum (x_i - \bar{x})^3}{(n - 1)(n - 2)s^3}$\\
    Skosnosc to miara symetrii/asymetrii | minusowa - lewo | dodatnia - prawo
    W naszym wypadku $\text{SKE} = \textbf{0.7}$, co oznacza, że prawe ramię wykresu jest wydłużone.
    
    \item[$\bullet$] \textbf{Kurtoza} liczona wzorem $\text{K} = \frac{m^4}{s^4} - 3$\\
    Kurtoza okresla intensywnosc wystyepowania wartosci skrajnych (w ogonach)
    w naszym przypadku wynosząca \textbf{3.69} wskazuje na sporą intensywność wartości skrajnych.
    
    \item[$\bullet$] \textbf{Odchylenie standardowe} liczone wzorem $s = \sqrt{\frac{1}{n-1} \sum_{i=1}^{n} (x_i - \bar{x})^2}$\\
    Mierzy rozproszenie zbioru według danych względem średniej, w naszym przypadku wynosi \textbf{0.6}.
    
    \item[$\bullet$] \textbf{Wariancja} liczona wzorem $\text{Var}(X) = \frac{1}{n-1} \sum_{i=1}^{n} (x_i - \bar{x})^2$\\
    To średnia z kwadratów odchyleń każdej wartości od średniej arytmetycznej zbioru danych w naszym przypadku 0.39

    \item[$\bullet$] \textbf{Srednia} $\bar{x} = \frac{1}{n} \sum_{i=1}^{n} x_i$\\
    Srednia cena na przestrzeni czasu to \textbf{11.18}

\end{itemize}

\hrulefill

\textbf{DOPASOWANIE GESTOSCI}

\Do do dopasaowania uzyjemy trzech funkcji rozkladu: \textbf{Normalnego}, \textbf{log-normalnego}, \textbf{Gamma}

\begin{itemize}
    \item[$\bullet$] \textbf{Wzor na gęstosc rozkladu Normalnego} $\f(x|\mu,\sigma^2) = \frac{1}{\sqrt{2\pi\sigma^2}} e^{-\frac{(x-\mu)^2}{2\sigma^2}}$

    \item[$\bullet$] \textbf{Wzor na gestosc rozkladu Log-Normalnego} $\f(x|\mu,\sigma) = \frac{1}{x\sigma\sqrt{2\pi}} e^{-\frac{(\ln x - \mu)^2}{2\sigma^2}}$

    \item[$\bullet$] \textbf{Wzor na gestosc rozkladu Gamma} $\f(x|k,\theta) = \frac{1}{\Gamma(k)\theta^k} x^{k-1} e^{-\frac{x}{\theta}}$
\end{itemize}


\begin{figure}
    \centering
    \includegraphics[width=1\linewidth]{Rplot02.png}
    \caption{\textbf{Wykres porównujący Empiryczne i Teoretyczne dystrybuanty}}
    \label{fig:enter-label1}
\end{figure}

\begin{figure}
    \centering
    \includegraphics[width=1\linewidth]{Rplot03.png}
    \caption{\textbf{Wykres porównujący kwantyle empiryczne i teoretyczne}}
    \label{fig:enter-label2}
\end{figure}

\begin{figure}
    \centering
    \includegraphics[width=1\linewidth]{Rplot05.png}
    \caption{\textbf{Wykres porównujący na gęstości}}
    \label{fig:enter-label3}
\end{figure}
\vspace{12pt}

\begin{itemize}
    \item[$\bullet$] \textbf{Wzor na test Kolmogorowa-Smirnov $\ D = \sup|F_n(x) - F(x)|$}

    \item[$\bullet$] \textbf{Wzor na test Cramera-von Misesa $\ W^2 = \int [F_n(x) - F(x)]^2 \, dF(x)$}

    \item[$\bullet$] \textbf{Wzor na test Anderson-Darling $\ A^2 = -n - \sum_{i=1}^{n} \frac{2i-1}{n} \left[ \ln(F(X_i)) + \ln(1 - F(X_{n+1-i})) \right]$}
\end{itemize}

\vspace{12pt}

\begin{table}[htbp]
    \centering
    \caption{\textbf{testy Goddes of Statistic}}
    \begin{tabular}{|c|c|c|c|}
        \hline
        \textbf{Statistic} & \textbf{norm} & \textbf{lnorm} & \textbf{gamma} \\
        \hline
        Kolmogorov-Smirnov & 0.1012138 & 0.09105258 & 0.09436892 \\
        Cramer-von Mises & 0.3689378 & 0.25513733 & 0.28957664 \\
        Anderson-Darling & 2.3711022 & 1.66370275 & 1.87978109 \\
        \hline
        \textbf{Information Criterion} & & & \\
        Akaike's & 474.6259 & 465.7726 & 468.4648 \\
        Bayesian & 481.6447 & 472.7914 & 475.4836 \\
        \hline
    \end{tabular}
    \label{tab:my-table}
\end{table}

\vspace{12pt}

\hrulefill

\textbf{Jak dobrac odpowiedni rozklad z testu?}\\wybieramy go poprzez wybieranie \textbf{najnizszej} wartosci wsrod testow
w naszym przypadku 5/0 najnizsza wartosc ma test dla rozkladu Log-normalnego.

\hrulefill

\vspace{12pt}

\textbf{Kolmogorov} - Smirnov sprawdza w najdalsza odleglolsc w jednym punkcie 

\vspace{12pt}

\textbf{Cramer - von mises} oblicza dystrybuante w kazdym punkcie z probki danych pozniej oblicza dystrubuante dla podanej funkcji (gamma,lnorm,norm) w naszym przypadku 
pozniej liczy kwadrat roznicy tych dystrybuant

\vspace{12pt}

\textbf{Andreson - Darling} suma kwadratow roznicy pomiedzy teoretyczna dystrybuanta a empiryczna 

\hrulefill

\textbf{Testowanie rownosci rozkladow}

\vspace{12pt}

\textbf{Wynik testu Monte carlo na histogramie} gdzie zaznaczony punkt to wynik dla oryginalnych danych.

\vspace{12pt}

\textbf{obliczamy p-value wzorem:} $\text{p-value} = \frac{\text{Liczba wartości w } Dln \text{ większych niż } dn_ln}{N}$

\vspace{12pt}

\textbf{Nasze p-value:} 0.0294

\vspace{12pt}

\textbf{przyjmujemy poziom istotnosci alpha }= 0.05

\vspace{12pt}

\text Po porownaniu czy \textbf{p-value $\leq$ alpha} hipoteze \textbf{odrzucamy} bo \textbf{p-value} jest mniejsza od poziomu istotnosci \textbf{alpha}

\begin{figure}[m]
    \centering
    \includegraphics[width=1\linewidth]{Rplot6.png}
    \caption{Histogram wynikow z testu Monte Carlo}
    \label{fig:enter-label}
\end{figure}

\end{document}
